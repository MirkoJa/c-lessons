%% Nothing to modify here.
%% make sure to include this before anything else

\documentclass[10pt]{beamer}
\usetheme{Szeged}
\usenavigationsymbolstemplate{}
\usepackage{tgcursor}

% packages
\usepackage{color}
\usepackage{listings}

% color definitions
\definecolor{mygreen}{rgb}{0,0.6,0}
\definecolor{mygray}{rgb}{0.5,0.5,0.5}
\definecolor{mymauve}{rgb}{0.58,0,0.82}

% re-format the title frame page
\makeatletter
\def\supertitle#1{\gdef\@supertitle{#1}}%
\setbeamertemplate{title page}
{
  \vbox{}
  \vfill
  \begin{centering}
  \begin{beamercolorbox}[sep=8pt,center]{title}
      \usebeamerfont{supertitle}\@supertitle
   \end{beamercolorbox}
    \begin{beamercolorbox}[sep=8pt,center]{title}
      \usebeamerfont{title}\inserttitle\par%
      \ifx\insertsubtitle\@empty%
      \else%
        \vskip0.25em%
        {\usebeamerfont{subtitle}\usebeamercolor[fg]{subtitle}\insertsubtitle\par}%
      \fi%     
    \end{beamercolorbox}%
    \vskip1em\par
    \begin{beamercolorbox}[sep=8pt,center]{author}
      \usebeamerfont{author}\insertauthor
    \end{beamercolorbox}
    \begin{beamercolorbox}[sep=8pt,center]{institute}
      \usebeamerfont{institute}\insertinstitute
    \end{beamercolorbox}
    \begin{beamercolorbox}[sep=8pt,center]{date}
      \usebeamerfont{date}\insertdate
    \end{beamercolorbox}\vskip0.5em
    {\usebeamercolor[fg]{titlegraphic}\inserttitlegraphic\par}
  \end{centering}
  \vfill
}
\makeatother

% insert frame number
\expandafter\def\expandafter\insertshorttitle\expandafter{%
      \insertshorttitle\hfill%
\insertframenumber\,/\,\inserttotalframenumber}

% preset-listing options
\lstset{
  backgroundcolor=\color{white},   
  % choose the background color; 
  % you must add \usepackage{color} or \usepackage{xcolor}
  basicstyle=\footnotesize,
  % the size of the fonts that are used for the code
  breakatwhitespace=false,         
  % sets if automatic breaks should only happen at whitespace
  breaklines=true,                 % sets automatic line breaking
  captionpos=b,                    % sets the caption-position to bottom
  commentstyle=\color{mygreen},    % comment style
  % deletekeywords={...},            
  % if you want to delete keywords from the given language
  extendedchars=true,              
  % lets you use non-ASCII characters; 
  % for 8-bits encodings only, does not work with UTF-8
  frame=single,                    % adds a frame around the code
  keepspaces=true,                 
  % keeps spaces in text, 
  % useful for keeping indentation of code 
  % (possibly needs columns=flexible)
  keywordstyle=\color{blue},       % keyword style
  % morekeywords={*,...},            
  % if you want to add more keywords to the set
  numbers=left,                    
  % where to put the line-numbers; possible values are (none, left, right)
  numbersep=5pt,                   
  % how far the line-numbers are from the code
  numberstyle=\tiny\color{mygray}, 
  % the style that is used for the line-numbers
  rulecolor=\color{black},         
  % if not set, the frame-color may be changed on line-breaks 
  % within not-black text (e.g. comments (green here))
  stepnumber=1,                    
  % the step between two line-numbers. 
  % If it's 1, each line will be numbered
  stringstyle=\color{mymauve},     % string literal style
  tabsize=4,                       % sets default tabsize to 4 spaces
  title=\lstname                   
  % show the filename of files included with \lstinputlisting; 
  % also try caption instead of title
}

% macro for code inclusion
\newcommand{\includecode}[2][c]{
	\lstinputlisting[caption=#2, style=custom#1]{#2}
}
	% nothing to do here
\usepackage[english]{babel}

\usepackage[utf8]{inputenc}

\newcommand{\course}{
	C advanced
}

\author{
	Richard Mörbitz,
	Manuel Thieme
}

\lstset{
	language = C,
	showspaces = false,
	showtabs = false,
	showstringspaces = false,
	escapechar = @,
	belowskip=-1.5em
}

\def\ContinueLineNumber{\lstset{firstnumber=last}}
\def\StartLineAt#1{\lstset{firstnumber=#1}}
\let\numberLineAt\StartLineAt

\makeatletter
\def\beamer@verbatimreadframe{%
	\begingroup%
	\let\do\beamer@makeinnocent\dospecials%
	\count@=127%
	\@whilenum\count@<255 \do{%
		\advance\count@ by 1%
		\catcode\count@=11%
	}%
	\beamer@makeinnocent\^^L% and whatever other special cases
	\beamer@makeinnocent\^^I% <-- PATCH: allows tabs to be written to temp file
	\endlinechar`\^^M \catcode`\^^M=12%
	\@ifnextchar\bgroup{\afterassignment\beamer@specialprocessframefirstline\let\beamer@temp=}{\beamer@processframefirstline}}%
\makeatother

% meta-information
\newcommand{\topic}{
	Debugging
}

\hypersetup{
	colorlinks=true,
	linkcolor=blue,
	urlcolor=blue,
}

% nothing to do here
\title{\topic}
\supertitle{\course}
\date{}

% the actual document
\begin{document}

\maketitle

\begin{frame}{Contents}
	\tableofcontents
\end{frame}

\begin{frame}{Getting started}
	The slides are at \href{https://fsr.github.io/c-lessons/materials.html}{fsr.github.io/c-lessons/materials.html}\\
	\bigskip
	There will be tasks! You can find them at \href{http://fsr.github.io/c-lessons/}{fsr.github.io/c-lessons}\\
	\bigskip
	If you have questions, use the auditorium group: \href{https://auditorium.inf.tu-dresden.de/de/groups/110804109}{https://auditorium.inf.tu-dresden.de/de/groups/110804109}\\
	\bigskip
	In case of big trouble, write an e-mail to your tutor.\\
	\bigskip
	\bigskip
	*** new only for a limited time ***\\
	Hackerspace every foo from bar to foobar in room biz.
\end{frame}

\section{Bugs}
\subsection{}
\begin{frame}{It's not a bug...}
	There are different kinds of errors.
	\begin{itemize}
		\item Compiletime errors
		\item Runtime errors (\textit{bugs})
	\end{itemize}\ \\\ \\
	\textit{Compiletime errors} are easily handable since the compiler shows you where to fix them.\\\ \\
	\textit{Bugs} on the other hand are harder to find because you have no idea where to look for them.
\end{frame}
\begin{frame}{... it's a feature.}
	Bugs can appear due to different reasons
	\begin{itemize}
		\item Variable overflow
		\item Division by zero
		\item Infinite loops / recursions
		\item Range excess
		\item Segmentation fault
		\item Dereferencing \textit{NULL pointers}
		\item ...
	\end{itemize}
\end{frame}
\begin{frame}{The dungeon}
	We prepared a little ASCII dungeon.\\
	You can find it in the repository (\url{https://github.com/fsr/c-slides}) in folder \textit{materials/1\_before/}\\
	\begin{itemize}
		\item Look at the code and try to understand what should happen.
		\item If you find mistakes, please leave them. We'll fix them later.
		\\\ 
		\item Compile it (with \textit{-std=c99}).
		\\\ 
		\item And now run it.
		\pause
		\item Try to fix all the mistakes using compiler flags.
	\end{itemize}
\end{frame}

\section{GDB}
\subsection{}
\begin{frame}[fragile]{The \textbf{G}NU \textbf{D}e\textbf{B}ugger}
	There are tools helping with bugs, called debuggers. GDB is one of them.\\
	\bigskip
	To use it
	\begin{itemize}
		\item You have to install the package \textit{gdb}\\
		\item You have to compile your program with the \textit{-g} flag
		\begin{lstlisting}[numbers=none]
$ gcc -g main.c
\end{lstlisting}
		\item After that you can start your program with gdb:
		\begin{lstlisting}[numbers=none]
$ gdb a.out
\end{lstlisting}
	\end{itemize}
\end{frame}
\begin{frame}{Commands}
	\begin{itemize}
		\item If you started gdb without a file you can load it with \textbf{file} \textit{file\_name}.
		\item Use \textbf{r[un]} to execute the program with gdb.\\
		You should begin with that. It will give you further information about the crash.
		\item You can set an arbitrary amount of breakpoints with \textbf{b[reak]} \textit{line\_number} or \textbf{b[reak]} \textit{function\_name}.\\
		Begin with a breakpoint at the point the program crashes.
		\item Print values with \textbf{p[rint]} \textit{identifier}.
		\item Use \textbf{w[atch]} \textit{identifier} to break and print a variable when it's changed.
	\end{itemize}
\end{frame}
\begin{frame}{Once you're at a breakpoint}
	\begin{itemize}
		\item Use \textbf{n[ext]} to execute the next program line only.
		\item \textbf{s[tep]} executes the next instruction.
		\item You can jump to the next breakpoint with \textbf{c[ontinue]}.
		\item To see how you have come to this point in the program flow, type \textbf{backtrace} or \textbf{bt}.\\
		This shows you all functions you called to come there.
		\\\ \\
		\item By only hitting the \textit{return key}, you repeat the last entered  command.
	\end{itemize}
	\ \\\ \\
	GDB is much more mighty than this few commands, but that should be sufficient to solve your final quest.
\end{frame}
\begin{frame}{Now it's up to you}
	\begin{itemize}
		\item Find and fix all Bugs in the dungeon.\\\ \\
		\begin{tabular}{|l|l|}
			\hline
			\textbf{file} & load program\\\hline
			\textbf{r[un]} & execute program\\\hline
			\textbf{b[reak]} & set breakpoint\\\hline
			\textbf{p[rint]} & print variable\\\hline
			\textbf{w[atch]} & break and print variable when it changes\\\hline
			\textbf{n[ext]} & execute next line and break\\\hline
			\textbf{s[tep]} & execute next instruction and break\\\hline
			\textbf{c[ontinue]} & execute until next breakpoint\\\hline
			\textbf{backtrace} / \textbf{bt} & How did I end up here?\\\hline
		\end{tabular}
	\end{itemize}
\end{frame}
% nothing to do from here on
\end{document}
