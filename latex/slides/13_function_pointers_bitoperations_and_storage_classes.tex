%% Nothing to modify here.
%% make sure to include this before anything else

\documentclass[10pt]{beamer}
\usetheme{Szeged}
\usenavigationsymbolstemplate{}
\usepackage{tgcursor}

% packages
\usepackage{color}
\usepackage{listings}

% color definitions
\definecolor{mygreen}{rgb}{0,0.6,0}
\definecolor{mygray}{rgb}{0.5,0.5,0.5}
\definecolor{mymauve}{rgb}{0.58,0,0.82}

% re-format the title frame page
\makeatletter
\def\supertitle#1{\gdef\@supertitle{#1}}%
\setbeamertemplate{title page}
{
  \vbox{}
  \vfill
  \begin{centering}
  \begin{beamercolorbox}[sep=8pt,center]{title}
      \usebeamerfont{supertitle}\@supertitle
   \end{beamercolorbox}
    \begin{beamercolorbox}[sep=8pt,center]{title}
      \usebeamerfont{title}\inserttitle\par%
      \ifx\insertsubtitle\@empty%
      \else%
        \vskip0.25em%
        {\usebeamerfont{subtitle}\usebeamercolor[fg]{subtitle}\insertsubtitle\par}%
      \fi%     
    \end{beamercolorbox}%
    \vskip1em\par
    \begin{beamercolorbox}[sep=8pt,center]{author}
      \usebeamerfont{author}\insertauthor
    \end{beamercolorbox}
    \begin{beamercolorbox}[sep=8pt,center]{institute}
      \usebeamerfont{institute}\insertinstitute
    \end{beamercolorbox}
    \begin{beamercolorbox}[sep=8pt,center]{date}
      \usebeamerfont{date}\insertdate
    \end{beamercolorbox}\vskip0.5em
    {\usebeamercolor[fg]{titlegraphic}\inserttitlegraphic\par}
  \end{centering}
  \vfill
}
\makeatother

% insert frame number
\expandafter\def\expandafter\insertshorttitle\expandafter{%
      \insertshorttitle\hfill%
\insertframenumber\,/\,\inserttotalframenumber}

% preset-listing options
\lstset{
  backgroundcolor=\color{white},   
  % choose the background color; 
  % you must add \usepackage{color} or \usepackage{xcolor}
  basicstyle=\footnotesize,
  % the size of the fonts that are used for the code
  breakatwhitespace=false,         
  % sets if automatic breaks should only happen at whitespace
  breaklines=true,                 % sets automatic line breaking
  captionpos=b,                    % sets the caption-position to bottom
  commentstyle=\color{mygreen},    % comment style
  % deletekeywords={...},            
  % if you want to delete keywords from the given language
  extendedchars=true,              
  % lets you use non-ASCII characters; 
  % for 8-bits encodings only, does not work with UTF-8
  frame=single,                    % adds a frame around the code
  keepspaces=true,                 
  % keeps spaces in text, 
  % useful for keeping indentation of code 
  % (possibly needs columns=flexible)
  keywordstyle=\color{blue},       % keyword style
  % morekeywords={*,...},            
  % if you want to add more keywords to the set
  numbers=left,                    
  % where to put the line-numbers; possible values are (none, left, right)
  numbersep=5pt,                   
  % how far the line-numbers are from the code
  numberstyle=\tiny\color{mygray}, 
  % the style that is used for the line-numbers
  rulecolor=\color{black},         
  % if not set, the frame-color may be changed on line-breaks 
  % within not-black text (e.g. comments (green here))
  stepnumber=1,                    
  % the step between two line-numbers. 
  % If it's 1, each line will be numbered
  stringstyle=\color{mymauve},     % string literal style
  tabsize=4,                       % sets default tabsize to 4 spaces
  title=\lstname                   
  % show the filename of files included with \lstinputlisting; 
  % also try caption instead of title
}

% macro for code inclusion
\newcommand{\includecode}[2][c]{
	\lstinputlisting[caption=#2, style=custom#1]{#2}
}
	% nothing to do here
\usepackage[english]{babel}

\usepackage[utf8]{inputenc}

\newcommand{\course}{
	C advanced
}

\author{
	Richard Mörbitz,
	Manuel Thieme
}

\lstset{
	language = C,
	showspaces = false,
	showtabs = false,
	showstringspaces = false,
	escapechar = @,
	belowskip=-1.5em
}

\def\ContinueLineNumber{\lstset{firstnumber=last}}
\def\StartLineAt#1{\lstset{firstnumber=#1}}
\let\numberLineAt\StartLineAt

\makeatletter
\def\beamer@verbatimreadframe{%
	\begingroup%
	\let\do\beamer@makeinnocent\dospecials%
	\count@=127%
	\@whilenum\count@<255 \do{%
		\advance\count@ by 1%
		\catcode\count@=11%
	}%
	\beamer@makeinnocent\^^L% and whatever other special cases
	\beamer@makeinnocent\^^I% <-- PATCH: allows tabs to be written to temp file
	\endlinechar`\^^M \catcode`\^^M=12%
	\@ifnextchar\bgroup{\afterassignment\beamer@specialprocessframefirstline\let\beamer@temp=}{\beamer@processframefirstline}}%
\makeatother % TODO modify this if you have not already done so

% meta-information
\newcommand{\topic}{
	Bit operations
}

% nothing to do here
\title{\topic}
\supertitle{\course}
\date{}

% the actual document
\begin{document}

\maketitle

\begin{frame}{Contents}
	\tableofcontents
\end{frame}

\section{Bit operations}
\subsection{}

\begin{frame}{A little bit of logic}
	As you know, all data is stored as \textit{binary numbers} - sequences of 0 and 1.\\
	In C, you can operate on this bit layer by using the following \textit{logical bit} and \textit{shift} operators:\bigskip
	
	\begin{tabular}{|c|c|l|}
																						  	  \hline
		\textbf{Symbol} 	& \textbf{Operation} 	& \textbf{Example} 							\\\hline
		$|$					& logical or				& $0110\ |\ 0101 == 0111$ 		\\\hline
		$\&$ 				& logical and 				& $0110\ \&\ 0101 == 0100$ 	\\\hline
		\textasciicircum				& logical xor 				& $0110\ $\textasciicircum\ $0101 == 0011$ 	\\\hline
		\textasciitilde			& logical negation 			& \textasciitilde $0110 == 1001$	\\\hline
		$<<$ 			& shift to the left 			& $0110\ <<\ 2 == 011000$ 	\\\hline
		$>>$ 		& shift to the right 				& $0110\ >>\ 2 == 0001$ 			\\\hline
	\end{tabular}
	
\end{frame}

\begin{frame}[fragile]{Computational arithmetics}
	With bit operations, some mathematical tasks can be solved more efficiently:
	\begin{itemize}
		\item Multiplying/dividing by $2^n$ is equivalent to a shift by $n$ bits
		\begin{lstlisting}
5 * 8 == 5 << 3;
60 / 4 == 60 >> 2;
\end{lstlisting}
		\item Instead of $\%\ 2^n$ you can use $\&\ 2^{n-1}$ 
		\begin{lstlisting}
22 % 2 == 22 & 1;
24 % 16 == 24 & 15;
\end{lstlisting}
	\end{itemize}\bigskip
	Be aware of the fact that the readability of your code will suffer from that. Most of these optimisations are done by the compiler anyway.
\end{frame}
\begin{frame}[fragile]{Masking}
	$x\ |\ mask$ sets all bits in $x$ that are $1$ in $mask$.\\
	\begin{lstlisting}
'A' | 32;	/* Let a capital letter be small */
\end{lstlisting}\bigskip
	$x\ \&\ mask$ deletes all bits in $x$ that are $0$ in $mask$.\\
	\begin{lstlisting}
'a' & ~32;	/* Let a small letter be capital */
\end{lstlisting}\bigskip
	$x\ $\textasciicircum\ $mask$ inverts all bits in $x$ that are $1$ in $mask$.\\
	\begin{lstlisting}
'a' ^ 32;	/* "Toggle" a letter */
\end{lstlisting}
\end{frame}

\begin{frame}[fragile]{Bit fields}
	Although it may seem efficient to use each bit of a number to store information in it, it will become nasty to access all the values by:\\
	$x\ \&\ 1$, $x\ \&\ 2$, $x\ \&\ 4$, \dots all the way up to $2^{sizeof\ int - 1}$\\
	\bigskip
	For this particular reason, C offers \textit{bit fields} like the following:
	\begin{lstlisting}
struct traffic_light {
	int red		: 1;
	int yellow	: 1;
	int green	: 1;
	int			: 5;	/* not in use */
};
\end{lstlisting}
	The members of bit fields can be accessed as if they were members of ordinary \textit{struct}s.

\end{frame}

% nothing to do from here on
\end{document}
