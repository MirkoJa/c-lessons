%% Nothing to modify here.
%% make sure to include this before anything else

\documentclass[10pt]{beamer}
\usetheme{Szeged}
\usenavigationsymbolstemplate{}
\usepackage{tgcursor}

% packages
\usepackage{color}
\usepackage{listings}

% color definitions
\definecolor{mygreen}{rgb}{0,0.6,0}
\definecolor{mygray}{rgb}{0.5,0.5,0.5}
\definecolor{mymauve}{rgb}{0.58,0,0.82}

% re-format the title frame page
\makeatletter
\def\supertitle#1{\gdef\@supertitle{#1}}%
\setbeamertemplate{title page}
{
  \vbox{}
  \vfill
  \begin{centering}
  \begin{beamercolorbox}[sep=8pt,center]{title}
      \usebeamerfont{supertitle}\@supertitle
   \end{beamercolorbox}
    \begin{beamercolorbox}[sep=8pt,center]{title}
      \usebeamerfont{title}\inserttitle\par%
      \ifx\insertsubtitle\@empty%
      \else%
        \vskip0.25em%
        {\usebeamerfont{subtitle}\usebeamercolor[fg]{subtitle}\insertsubtitle\par}%
      \fi%     
    \end{beamercolorbox}%
    \vskip1em\par
    \begin{beamercolorbox}[sep=8pt,center]{author}
      \usebeamerfont{author}\insertauthor
    \end{beamercolorbox}
    \begin{beamercolorbox}[sep=8pt,center]{institute}
      \usebeamerfont{institute}\insertinstitute
    \end{beamercolorbox}
    \begin{beamercolorbox}[sep=8pt,center]{date}
      \usebeamerfont{date}\insertdate
    \end{beamercolorbox}\vskip0.5em
    {\usebeamercolor[fg]{titlegraphic}\inserttitlegraphic\par}
  \end{centering}
  \vfill
}
\makeatother

% insert frame number
\expandafter\def\expandafter\insertshorttitle\expandafter{%
      \insertshorttitle\hfill%
\insertframenumber\,/\,\inserttotalframenumber}

% preset-listing options
\lstset{
  backgroundcolor=\color{white},   
  % choose the background color; 
  % you must add \usepackage{color} or \usepackage{xcolor}
  basicstyle=\footnotesize,
  % the size of the fonts that are used for the code
  breakatwhitespace=false,         
  % sets if automatic breaks should only happen at whitespace
  breaklines=true,                 % sets automatic line breaking
  captionpos=b,                    % sets the caption-position to bottom
  commentstyle=\color{mygreen},    % comment style
  % deletekeywords={...},            
  % if you want to delete keywords from the given language
  extendedchars=true,              
  % lets you use non-ASCII characters; 
  % for 8-bits encodings only, does not work with UTF-8
  frame=single,                    % adds a frame around the code
  keepspaces=true,                 
  % keeps spaces in text, 
  % useful for keeping indentation of code 
  % (possibly needs columns=flexible)
  keywordstyle=\color{blue},       % keyword style
  % morekeywords={*,...},            
  % if you want to add more keywords to the set
  numbers=left,                    
  % where to put the line-numbers; possible values are (none, left, right)
  numbersep=5pt,                   
  % how far the line-numbers are from the code
  numberstyle=\tiny\color{mygray}, 
  % the style that is used for the line-numbers
  rulecolor=\color{black},         
  % if not set, the frame-color may be changed on line-breaks 
  % within not-black text (e.g. comments (green here))
  stepnumber=1,                    
  % the step between two line-numbers. 
  % If it's 1, each line will be numbered
  stringstyle=\color{mymauve},     % string literal style
  tabsize=4,                       % sets default tabsize to 4 spaces
  title=\lstname                   
  % show the filename of files included with \lstinputlisting; 
  % also try caption instead of title
}

% macro for code inclusion
\newcommand{\includecode}[2][c]{
	\lstinputlisting[caption=#2, style=custom#1]{#2}
}
	% nothing to do here
\usepackage[english]{babel}

\usepackage[utf8]{inputenc}

\newcommand{\course}{
	C advanced
}

\author{
	Richard Mörbitz,
	Manuel Thieme
}

\lstset{
	language = C,
	showspaces = false,
	showtabs = false,
	showstringspaces = false,
	escapechar = @,
	belowskip=-1.5em
}

\def\ContinueLineNumber{\lstset{firstnumber=last}}
\def\StartLineAt#1{\lstset{firstnumber=#1}}
\let\numberLineAt\StartLineAt

\makeatletter
\def\beamer@verbatimreadframe{%
	\begingroup%
	\let\do\beamer@makeinnocent\dospecials%
	\count@=127%
	\@whilenum\count@<255 \do{%
		\advance\count@ by 1%
		\catcode\count@=11%
	}%
	\beamer@makeinnocent\^^L% and whatever other special cases
	\beamer@makeinnocent\^^I% <-- PATCH: allows tabs to be written to temp file
	\endlinechar`\^^M \catcode`\^^M=12%
	\@ifnextchar\bgroup{\afterassignment\beamer@specialprocessframefirstline\let\beamer@temp=}{\beamer@processframefirstline}}%
\makeatother % TODO modify this if you have not already done so

% meta-information
\newcommand{\topic}{%
    Network Programming
}

% nothing to do here
\title{\topic}
\supertitle{\course}
\date{}

% the actual document
\begin{document}

\maketitle

\begin{frame}{Contents}
	\tableofcontents
\end{frame}

\section{Network Protocols}
\subsection{}

\begin{frame}{Osi Reference Model}
    % OSI model, Pfeil auf Layer 4
\end{frame}

\begin{frame}{Transport Protocols}
    TCP:
    \begin{itemize}
        \item bissl Foo
    \end{itemize}
    UDP:
    \begin{itemize}
        \item bissl Bar
    \end{itemize}
    % vielleicht auch keine zwei itemize, sondern was schöneres
\end{frame}

\section{Socket Programming}
\subsection{}

\begin{frame}[fragile]{Sockets}
    Sockets are abstractions for connection endpoints to be used by processes.
    Both the server and the client process have a socket which they use to
    send data to each other.\\
    \bigskip
    Sockets are platform-dependend, but the system call interface is similar:
    \begin{description}
        \item[Unix] file descriptors (\lstinline{int})
        \item[Windows] handles for kernel objects
            (\lstinline[morekeywords={*,SOCKET}]{SOCKET})
    \end{description}
    \bigskip
    You will also have to include different headers:
    \begin{lstlisting}[numbers=none]
// Unix
#include <sys/socket.h>
// Windows
#include <windows.h>
\end{lstlisting}
    % Unix: files (int - file descriptor), Windows: objects (SOCKET - handle)
    % und die beiden Header
\end{frame}

% Dann noch jeweils ein Frame zu folgenden Funktionen mit kurzer Erklärung
% sowie Windows- und Unix-Prototyp
% Dabei die Argumente und einsetzbaren Konstanten erklären
% - socket()
% -- hier evtl. noch ein Frame mit Erklärung der Netzwerkadressrechnung
% - close[socket]()

\begin{frame}[fragile]{\texttt{connect()}}
    Connect a socket to another via the network.
    \begin{lstlisting}[numbers=none,morekeywords={*,SOCKET,socklen_t,sockaddr}]
// Unix
int connect(int socket, const struct sockaddr *address,
         socklen_t address_len);
// Windows
int connect(SOCKET socket, const struct sockaddr *address,
         int address_len);
\end{lstlisting}
    \begin{description}
        \item[socket] Socket to be connected
        \item[address] Structure containing target IP address and port
        \item[address\_len] Size of *address in memory
        \item[return value] Exit status (0 = success, -1 = failure)
    \end{description}
    \bigskip
    UDP sockets don't establish a connection $\rightarrow$ \texttt{connect()} is
    optional.
\end{frame}

\begin{frame}[fragile]{\texttt{bind()}}
    Bind an address to a socket.
    \begin{lstlisting}[numbers=none,morekeywords={*,SOCKET,socklen_t,sockaddr}]
// Unix
int bind(int socket, const struct sockaddr *address,
         socklen_t address_len);
// Windows
int bind(SOCKET socket, const struct sockaddr *address,
         int address_len);
\end{lstlisting}
    \begin{description}
        \item[socket] Socket to be bound
        \item[address] Structure containing IP address and port
        \item[address\_len] Size of *address in memory
        \item[return value] Exit status (0 = success, -1 = failure)
    \end{description}
    \bigskip
    Naming a socket is necessary for connections from the outside!
\end{frame}

\begin{frame}[fragile]{\texttt{listen()}}
    Enable listening for connections to a specific socket.
    \begin{lstlisting}[numbers=none,morekeywords={*,SOCKET}]
// Unix
int listen(int socket, int backlog);
// Windows
int listen(SOCKET socket, int backlog);
\end{lstlisting}
    \begin{description}
        \item[socket] Socket to put into listening mode
        \item[backlog] Hint for an upper bound of the number of outstanding
            connections in the listening queue of the socket
        \item[return value] Exit status (0 = success, -1 = failure)
    \end{description}
    \bigskip
    Calling \texttt{listen()} on a socket is necessary to accept incoming TCP
    connections on a server.
\end{frame}

\begin{frame}[fragile]{\texttt{accept()}}
    Accept a new connection on a socket.
    \begin{lstlisting}[numbers=none,morekeywords={*,SOCKET,sockaddr,socklen_t}]
// Unix
int accept(int socket, struct sockaddr *restrict address,
           socklen_t *restrict address_len);
// Windows
SOCKET accept(SOCKET socket, struct sockaddr *address,
              int *address_len);
\end{lstlisting}
    \begin{description}
        \item[socket] Listening socket
        \item[address] Where to store the address of the
            connecting socket
        \item[address\_len] Size of *address in memory
        \item[return value] Socket for the new connection on success, invalid
            descriptor otherwise
    \end{description}
    \bigskip
    By default, \texttt{accept()} blocks if the socket's connection queue is empty!
\end{frame}

% - send() / recv() / sendto() / recvfrom()}

% nothing to do from here on
\end{document}
