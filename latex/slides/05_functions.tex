%% Nothing to modify here.
%% make sure to include this before anything else

\documentclass[10pt]{beamer}
\usetheme{Szeged}
\usenavigationsymbolstemplate{}
\usepackage{tgcursor}

% packages
\usepackage{color}
\usepackage{listings}

% color definitions
\definecolor{mygreen}{rgb}{0,0.6,0}
\definecolor{mygray}{rgb}{0.5,0.5,0.5}
\definecolor{mymauve}{rgb}{0.58,0,0.82}

% re-format the title frame page
\makeatletter
\def\supertitle#1{\gdef\@supertitle{#1}}%
\setbeamertemplate{title page}
{
  \vbox{}
  \vfill
  \begin{centering}
  \begin{beamercolorbox}[sep=8pt,center]{title}
      \usebeamerfont{supertitle}\@supertitle
   \end{beamercolorbox}
    \begin{beamercolorbox}[sep=8pt,center]{title}
      \usebeamerfont{title}\inserttitle\par%
      \ifx\insertsubtitle\@empty%
      \else%
        \vskip0.25em%
        {\usebeamerfont{subtitle}\usebeamercolor[fg]{subtitle}\insertsubtitle\par}%
      \fi%     
    \end{beamercolorbox}%
    \vskip1em\par
    \begin{beamercolorbox}[sep=8pt,center]{author}
      \usebeamerfont{author}\insertauthor
    \end{beamercolorbox}
    \begin{beamercolorbox}[sep=8pt,center]{institute}
      \usebeamerfont{institute}\insertinstitute
    \end{beamercolorbox}
    \begin{beamercolorbox}[sep=8pt,center]{date}
      \usebeamerfont{date}\insertdate
    \end{beamercolorbox}\vskip0.5em
    {\usebeamercolor[fg]{titlegraphic}\inserttitlegraphic\par}
  \end{centering}
  \vfill
}
\makeatother

% insert frame number
\expandafter\def\expandafter\insertshorttitle\expandafter{%
      \insertshorttitle\hfill%
\insertframenumber\,/\,\inserttotalframenumber}

% preset-listing options
\lstset{
  backgroundcolor=\color{white},   
  % choose the background color; 
  % you must add \usepackage{color} or \usepackage{xcolor}
  basicstyle=\footnotesize,
  % the size of the fonts that are used for the code
  breakatwhitespace=false,         
  % sets if automatic breaks should only happen at whitespace
  breaklines=true,                 % sets automatic line breaking
  captionpos=b,                    % sets the caption-position to bottom
  commentstyle=\color{mygreen},    % comment style
  % deletekeywords={...},            
  % if you want to delete keywords from the given language
  extendedchars=true,              
  % lets you use non-ASCII characters; 
  % for 8-bits encodings only, does not work with UTF-8
  frame=single,                    % adds a frame around the code
  keepspaces=true,                 
  % keeps spaces in text, 
  % useful for keeping indentation of code 
  % (possibly needs columns=flexible)
  keywordstyle=\color{blue},       % keyword style
  % morekeywords={*,...},            
  % if you want to add more keywords to the set
  numbers=left,                    
  % where to put the line-numbers; possible values are (none, left, right)
  numbersep=5pt,                   
  % how far the line-numbers are from the code
  numberstyle=\tiny\color{mygray}, 
  % the style that is used for the line-numbers
  rulecolor=\color{black},         
  % if not set, the frame-color may be changed on line-breaks 
  % within not-black text (e.g. comments (green here))
  stepnumber=1,                    
  % the step between two line-numbers. 
  % If it's 1, each line will be numbered
  stringstyle=\color{mymauve},     % string literal style
  tabsize=4,                       % sets default tabsize to 4 spaces
  title=\lstname                   
  % show the filename of files included with \lstinputlisting; 
  % also try caption instead of title
}

% macro for code inclusion
\newcommand{\includecode}[2][c]{
	\lstinputlisting[caption=#2, style=custom#1]{#2}
}
  % nothing to do here
\input{c_introduction_info} % TODO modify this if you have not already done so

% meta-information
\newcommand {\topic}{
    Functions
}
\usepackage{tikz}
\usepackage[absolute,overlay]{textpos}

\setlength{\TPHorizModule}{1cm}
\setlength{\TPVertModule}{1cm}

\lstset{
  moredelim=**[is][\only<+(1)>{\color{red}}]{§}{§},
}

% nothing to do here
\title{\topic}
\supertitle{\course}
\date{}

% the actual document
\begin{document}

\maketitle

\begin{frame}{Contents}
    \tableofcontents
\end{frame}

\section{Using functions}
\subsection{}

\begin{frame}[fragile]{Remember the main \textit{function}?}
	\bigskip
	\begin{lstlisting}[numbers=none]
int main(void) {
	/* code happens */
	return 0;
}
\end{lstlisting}
\end{frame}

\begin{frame}[fragile]{Defining functions}
	\begin{columns}[T]
		\column{.33\textwidth}<2->
		data type of the returned value or \textit{void}, if nothing is returned
		\column{.33\textwidth}<3->
		unique name to refer the function, same rules as for variable identifiers
		\column{.34\textwidth}<4->
		parameter declarations, seperated by commas or \textit{void}, if there are none
	\end{columns}
	\begin{textblock}{7}(1.5,3.5)
		\begin{tikzpicture}[x=1.2cm,y=.5cm]
			\only<2>{\draw[red] (-.3,0) edge[out=270,in=90,->,shorten >=0ex] (0,-2);}
			\uncover<3->{\draw (-.3,0) edge[out=270,in=90,->,shorten >=0ex] (0,-2);}
			\only<3>{\draw[red] (3.5,0) edge[out=270,in=90,->,shorten >=0ex] (2.25,-2);}
			\uncover<4->{\draw (3.5,0) edge[out=270,in=90,->,shorten >=0ex] (2.25,-2);}
			\only<4>{\draw[red] (7,0) edge[out=270,in=90,->,shorten >=.0ex] (4.5,-2);}
			\uncover<5->{\draw (7,0) edge[out=270,in=90,->,shorten >=.0ex] (4.5,-2);}
		\end{tikzpicture}
	\end{textblock}
	\ \\\ \\\ \\
	\begin{lstlisting}[numbers=none,basicstyle=\itshape\small]
§return_type§ §identifier§(§argument_list§) {
	§function_body§
	return §expression§;
}
\end{lstlisting}
	\	\\\	\\\ \\
	\begin{columns}[T]
		\column{.5\textwidth}<5->
		just as in \textit{main()}, all statements are put in here
		\column{.5\textwidth}<6->
		value this function returns or empty, if the return value is \textit{void}
	\end{columns}
	\begin{textblock}{7}(.5,5.2)
		\begin{tikzpicture}
			\only<5>{\draw[red] (1.5,0) edge[out=90,in=180,->,shorten >=0ex] (0,2);}
			\uncover<6->{\draw (1.5,0) edge[out=90,in=180,->,shorten >=0ex] (0,2);}
			\only<6>{\draw[red] (8,0) edge[out=90,in=270,->,shorten >=0ex] (2.4,1.5);}
			\uncover<7->{\draw (8,0) edge[out=90,in=270,->,shorten >=0ex] (2.4,1.5);}
		\end{tikzpicture}
	\end{textblock}
\end{frame}

\begin{frame}[fragile]{Passing arguments}
	\begin{itemize}
		\item Each value is assigned to the parameter at the same position in the argument list (and therefore must have the same type)
	\end{itemize}		
	\begin{lstlisting}[basicstyle=\scriptsize]
#include <stdio.h>	
	
void shift_character(char character, unsigned offset) {
	printf("%c\n", (character + offset) % 255);
}

int random_number(void) {
	return 4;	// chosen by fair dice roll.
				// guaranteed to be random.
}

int main(void) {
	int offset = 10;
	shift_character('c', offset);
	printf("%d\n", random_number());
	return 0;
}
\end{lstlisting}
\end{frame}

\section{More on scopes}
\subsection{}
\begin{frame}[fragile]{Global variables}
	\begin{itemize}
		\item Variables defined outside any function
		\item Scope: from line of declaration to end of program
	\end{itemize}
	\begin{lstlisting}
int globe = 42;

void foo() {
	globe = 23;
}

int main(void) {
	printf("%d\n", globe);	/* Prints 42 */
	foo();
	printf("%d\n", globe);	/* Prints 23 */
	...
\end{lstlisting}
	Altering them in one function may have \textbf{side effects} on other functions $\rightarrow$ use them rarely.
\end{frame}
\begin{frame}[fragile]{Where not to call functions}
	Since a function's scope starts at the line of its definition, having two functions \textit{f()} and \textit{g()} calling each other is not possible:
	\begin{lstlisting}
void f(int i) {
	...
	g(42);	/* What is g? */
}

void g(int i) {
	...
	f(42);
}
\end{lstlisting}
	In that case, \textit{g()} is called outside its scope. Changing the order does not work either.
\end{frame}
\begin{frame}[fragile]{Prototypes}
	Like variables, functions can also be \textit{declared}:
	\begin{lstlisting}[numbers=none,basicstyle=\itshape\footnotesize]
return_type identifier(argument list);
\end{lstlisting}
	\begin{itemize}
		\item It's similar to a definition, just replace the function body by a \textit{;}
		\item Declared functions must also be defined any where in the program
		\item In the argument list, only types matter $\rightarrow$ identifiers \textbf{can} be left out
	\end{itemize}
	\begin{lstlisting}
void g(int i);	/* better do not leave the identifier out */

void f(int i) {
	...
	g(42);		/* Now a call of g() can be compiled */
}

void g(int i) {...}	/* g() definition, similar to f() */
\end{lstlisting}
\end{frame}
\begin{frame}[fragile]{Better program structure}
	To avoid problems like that above, it is a common practise to \textit{declare} all functions at the top of the file and define them below the main function:
	\begin{lstlisting}
void f(int i);
void g(int i);

int main(void) {
	...
}

void f(int i) {
	...
	g(42);
}

/* g() definition, similar to f() */
\end{lstlisting}
\end{frame}
\begin{frame}{Functions in functions}
	You \textbf{could} define functions in functions.\footnotemark
	
	\footnotetext[1]{Just saying.}
\end{frame}
\section{Recursion}
\subsection{}
\begin{frame}[fragile]{Recursive functions}
	\begin{itemize}
		\item Functions calling themselves
		\item Used to implement many mathematical algorithms
		\item Easy to think up, but they run slow
	\end{itemize} \ \\ \ \\
	Careful:
	\begin{lstlisting}
void foo() {
	foo();
}
\end{lstlisting}
	creates an infinite loop.\footnote{And, at some point, a program crash (\textit{stack overflow})} \\
	There must always be an \textit{exit condition} if using recursion!
\end{frame}
\begin{frame}[fragile]{Exponentiation}
As an example, take a look at this function calculating $base^{exponent}$:
	\begin{lstlisting}
int power(int base, int exponent) {
	if (exponent =@\,@= 0)
		return 1;
	return base * power(base, exponent - 1);
}
\end{lstlisting}
	\begin{itemize}
		\item $a^{0} = 1 \rightarrow$ \textit{power(a, 0}) just returns \textit{1}
		\item $a^{b} = a * a^{b-1} \rightarrow$ recursive call of \textit{power(a, b-1)}
	\end{itemize}
\end{frame}
% nothing to do from here on
\end{document}
