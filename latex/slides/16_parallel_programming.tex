%% Nothing to modify here.
%% make sure to include this before anything else

\documentclass[10pt]{beamer}
\usetheme{Szeged}
\usenavigationsymbolstemplate{}
\usepackage{tgcursor}

% packages
\usepackage{color}
\usepackage{listings}

% color definitions
\definecolor{mygreen}{rgb}{0,0.6,0}
\definecolor{mygray}{rgb}{0.5,0.5,0.5}
\definecolor{mymauve}{rgb}{0.58,0,0.82}

% re-format the title frame page
\makeatletter
\def\supertitle#1{\gdef\@supertitle{#1}}%
\setbeamertemplate{title page}
{
  \vbox{}
  \vfill
  \begin{centering}
  \begin{beamercolorbox}[sep=8pt,center]{title}
      \usebeamerfont{supertitle}\@supertitle
   \end{beamercolorbox}
    \begin{beamercolorbox}[sep=8pt,center]{title}
      \usebeamerfont{title}\inserttitle\par%
      \ifx\insertsubtitle\@empty%
      \else%
        \vskip0.25em%
        {\usebeamerfont{subtitle}\usebeamercolor[fg]{subtitle}\insertsubtitle\par}%
      \fi%     
    \end{beamercolorbox}%
    \vskip1em\par
    \begin{beamercolorbox}[sep=8pt,center]{author}
      \usebeamerfont{author}\insertauthor
    \end{beamercolorbox}
    \begin{beamercolorbox}[sep=8pt,center]{institute}
      \usebeamerfont{institute}\insertinstitute
    \end{beamercolorbox}
    \begin{beamercolorbox}[sep=8pt,center]{date}
      \usebeamerfont{date}\insertdate
    \end{beamercolorbox}\vskip0.5em
    {\usebeamercolor[fg]{titlegraphic}\inserttitlegraphic\par}
  \end{centering}
  \vfill
}
\makeatother

% insert frame number
\expandafter\def\expandafter\insertshorttitle\expandafter{%
      \insertshorttitle\hfill%
\insertframenumber\,/\,\inserttotalframenumber}

% preset-listing options
\lstset{
  backgroundcolor=\color{white},   
  % choose the background color; 
  % you must add \usepackage{color} or \usepackage{xcolor}
  basicstyle=\footnotesize,
  % the size of the fonts that are used for the code
  breakatwhitespace=false,         
  % sets if automatic breaks should only happen at whitespace
  breaklines=true,                 % sets automatic line breaking
  captionpos=b,                    % sets the caption-position to bottom
  commentstyle=\color{mygreen},    % comment style
  % deletekeywords={...},            
  % if you want to delete keywords from the given language
  extendedchars=true,              
  % lets you use non-ASCII characters; 
  % for 8-bits encodings only, does not work with UTF-8
  frame=single,                    % adds a frame around the code
  keepspaces=true,                 
  % keeps spaces in text, 
  % useful for keeping indentation of code 
  % (possibly needs columns=flexible)
  keywordstyle=\color{blue},       % keyword style
  % morekeywords={*,...},            
  % if you want to add more keywords to the set
  numbers=left,                    
  % where to put the line-numbers; possible values are (none, left, right)
  numbersep=5pt,                   
  % how far the line-numbers are from the code
  numberstyle=\tiny\color{mygray}, 
  % the style that is used for the line-numbers
  rulecolor=\color{black},         
  % if not set, the frame-color may be changed on line-breaks 
  % within not-black text (e.g. comments (green here))
  stepnumber=1,                    
  % the step between two line-numbers. 
  % If it's 1, each line will be numbered
  stringstyle=\color{mymauve},     % string literal style
  tabsize=4,                       % sets default tabsize to 4 spaces
  title=\lstname                   
  % show the filename of files included with \lstinputlisting; 
  % also try caption instead of title
}

% macro for code inclusion
\newcommand{\includecode}[2][c]{
	\lstinputlisting[caption=#2, style=custom#1]{#2}
}
	% nothing to do here
\usepackage[english]{babel}

\usepackage[utf8]{inputenc}

\newcommand{\course}{
	C advanced
}

\author{
	Richard Mörbitz,
	Manuel Thieme
}

\lstset{
	language = C,
	showspaces = false,
	showtabs = false,
	showstringspaces = false,
	escapechar = @,
	belowskip=-1.5em
}

\def\ContinueLineNumber{\lstset{firstnumber=last}}
\def\StartLineAt#1{\lstset{firstnumber=#1}}
\let\numberLineAt\StartLineAt

\makeatletter
\def\beamer@verbatimreadframe{%
	\begingroup%
	\let\do\beamer@makeinnocent\dospecials%
	\count@=127%
	\@whilenum\count@<255 \do{%
		\advance\count@ by 1%
		\catcode\count@=11%
	}%
	\beamer@makeinnocent\^^L% and whatever other special cases
	\beamer@makeinnocent\^^I% <-- PATCH: allows tabs to be written to temp file
	\endlinechar`\^^M \catcode`\^^M=12%
	\@ifnextchar\bgroup{\afterassignment\beamer@specialprocessframefirstline\let\beamer@temp=}{\beamer@processframefirstline}}%
\makeatother % TODO modify this if you have not already done so

% meta-information
\newcommand{\topic}{
	Parallel programming
}

% nothing to do here
\title{\topic}
\supertitle{\course}
\date{}

% the actual document
\begin{document}

\maketitle

\begin{frame}{Contents}
	\tableofcontents
\end{frame}

\section{Threads}
\subsection{}
\begin{frame}{Threads}
    By calling the \textit{main()} function of your program, the OS creates a thread.
    It is up to you to create more of them.\\
    \bigskip
    There is the system call \textit{fork()} create a new process. From this call on you have two completely independent processes of your program running. In the parent thread it returns the pid of the child process. In the child process it returns zero.
    
\end{frame}
\begin{frame}[fragile]{Use the fork}
    \begin{lstlisting}
#include <unistd.h>

int main(void) {
    pid_t pid = fork();

    if (pid @=@@=@ 0) {
        /* do stuff in child process */
    } else if (pid > 0) {
        /* do stuff in parent process */
    } else {
        /* fork failed */
        return 1;
    }
    return 0;
}
\end{lstlisting}
    See the man page for further information.
\end{frame}

\section{pthreads}
\subsection{}
\begin{frame}[fragile]{pthread\_create}
    The better way to implement threading are the P[osix] threads:
    \bigskip
    \begin{lstlisting}[numbers=none]
int pthread_create(pthread_t *thread,
                   const pthread_attr_t *attr,
                   void *(*start_routine) (void *),
                   void *arg);
\end{lstlisting}
    starts a new thread by calling \textit{start\_routine()}.
\end{frame}

\begin{frame}[fragile]{Hello Threads}
    \begin{lstlisting}
#include <pthread.h>
#include <stdio.h>

void *hello_thread(void *tid);

int main(void) {
    pthread_t threads[5];
    for(int i=0; i < 5; ++i) {
       if (pthread_create(&threads[i], NULL,
                          hello_thread, (void *)i))
          return 1;
    }
    return 0;
}

void *hello_thread(void *tid) {
    printf("Hello, I am thread %d\n", *tid);
    pthread_exit(NULL);
}
\end{lstlisting}
\end{frame}
\begin{frame}[fragile]{Ways to end a thread:}
    \begin{itemize}
        \item \textit{start\_routine} returns
        \item \textit{pthread\_exit} is called
        \begin{lstlisting}[numbers=none]
void pthread_exit(void *retval);
\end{lstlisting}
        \item \textit{pthread\_cancel} is called from another thread
        \begin{lstlisting}[numbers=none]
int pthread_cancel(pthread_t thread);
\end{lstlisting}
        \item \textit{exit} is called from somewhere
        \bigskip
        \item no explicit \textit{pthread\_exit} is called and main  finishes before \textit{start\_routine}
    \end{itemize}
\end{frame}
\begin{frame}[fragile]{Waiting for threads}
    To wait for a thread to finish, there is \textit{pthread\_join}
    \begin{lstlisting}[numbers=none]
int pthread_join(pthread_t thread, void **value_ptr);
\end{lstlisting}
   
    \bigskip
    There is the \textit{pthread\_attr\_t} structure you can pass to \textit{pthread\_create}. You have to initialise the structure first and destroy it later:
    \begin{lstlisting}[numbers=none]
int pthread_attr_init(pthread_attr_t *attr);
int pthread_attr_destroy(pthread_attr_t *attr);
\end{lstlisting}
    \bigskip
    After that you can set and get the detachstate:
    \begin{lstlisting}[numbers=none]
int pthread_attr_setdetachstate(pthread_attr_t *attr,
                                int detachstate);
int pthread_attr_getdetachstate(const pthread_attr_t *attr,
                                int *detachstate);
\end{lstlisting}
\end{frame}

\begin{frame}[fragile]{bigger, better, joinable}
    \begin{lstlisting}[basicstyle=\scriptsize]
int main(void) {
    pthread_t threads[5];
    pthread_attr_t attr;
    
    pthread_attr_init(&attr);
    pthread_attr_setdetachstate(&attr, PTHREAD_CREATE_JOINABLE);
    
    for(int i=0; i < 5; ++i) {
       if (pthread_create(&threads[i], &attr,
                          hello_thread, (void *)i))
          return 1;
    }
    pthread_attr_destroy(&attr);

    void *st;
    for(int i=0; i < 5; ++i) {
       if(pthread_join(thread[i], &st))
           return 1;
       printf("Thread %d finished with status %d\n", i, *st);
    }
    
    return 0;
}
\end{lstlisting}
\end{frame}

\section{Be careful}
\subsection{}
\begin{frame}[fragile]{Mutexes}
    Since we pass a pointer as \textit{arg} to \textit{pthread\_create} the value behind this pointer could be modified by multiple threads. To avoid race conditions, the pthread library provides mutexes.
    \begin{lstlisting}[numbers=none]
int pthread_mutex_destroy(pthread_mutex_t *mutex);
int pthread_mutex_init(pthread_mutex_t *restrict mutex,
                 const pthread_mutexattr_t *restrict attr);
pthread_mutex_t mutex = PTHREAD_MUTEX_INITIALIZER;
\end{lstlisting}
    \bigskip
    A Mutex is a datatyp that can be locked before and unlocked after accessing a variable.
    \begin{lstlisting}[numbers=none]
int pthread_mutex_lock(pthread_mutex_t *mutex);
int pthread_mutex_trylock(pthread_mutex_t *mutex);
int pthread_mutex_unlock(pthread_mutex_t *mutex);
\end{lstlisting}
\end{frame}

\begin{frame}[fragile]{Lock me if you can}
    \begin{lstlisting}[basicstyle=\tiny]
struct stuff {
    unsigned a;
    unsigned b;
}

struct stuff global = {1, 2};
pthread_mutex_t mutex;

void *thread_stuff(void *tid);
unsigned do_stuff(unsigned u);

int main(void) {
    pthread_t threads[5];
    pthread_mutex_init(&mutex, NULL);
    for (int i = 0; i < 5; ++i) {
        if (pthread_create(&thread[i], NULL, threadstuff, NULL))
            return 1;
    }
    
    pthread_mutex_destroy(&mutex)
    return 0;
}

void *thread_stuff(void *tid) {
    pthread_mutex_lock(mutex);
    unsigned a = do_stuff(global.a);
    global.b = a;
    pthread_mutex_unlock(mutex);
    
    pthread_exit(NULL);
}
\end{lstlisting}
\end{frame}

\begin{frame}[fragile]{Deadlocks incoming}
    Consider the following:
     \begin{lstlisting}[basicstyle=\scriptsize]
void *thread_1(void *tid) {
    pthread_mutex_lock(mutex_1);
    pthread_mutex_lock(mutex_2);
    /* stuff */
    pthread_mutex_unlock(mutex_1);
    pthread_mutex_unlock(mutex_2);
    
    pthread_exit(NULL);
}

void *thread_2(void *tid) {
    pthread_mutex_lock(mutex_2);
    pthread_mutex_lock(mutex_1);
    /* stuff */
    pthread_mutex_unlock(mutex_2);
    pthread_mutex_unlock(mutex_1);
    
    pthread_exit(NULL);
}
\end{lstlisting}
\end{frame}

% nothing to do from here on
\end{document}
