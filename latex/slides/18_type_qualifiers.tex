%% Nothing to modify here.
%% make sure to include this before anything else

\documentclass[10pt]{beamer}
\usetheme{Szeged}
\usenavigationsymbolstemplate{}
\usepackage{tgcursor}

% packages
\usepackage{color}
\usepackage{listings}

% color definitions
\definecolor{mygreen}{rgb}{0,0.6,0}
\definecolor{mygray}{rgb}{0.5,0.5,0.5}
\definecolor{mymauve}{rgb}{0.58,0,0.82}

% re-format the title frame page
\makeatletter
\def\supertitle#1{\gdef\@supertitle{#1}}%
\setbeamertemplate{title page}
{
  \vbox{}
  \vfill
  \begin{centering}
  \begin{beamercolorbox}[sep=8pt,center]{title}
      \usebeamerfont{supertitle}\@supertitle
   \end{beamercolorbox}
    \begin{beamercolorbox}[sep=8pt,center]{title}
      \usebeamerfont{title}\inserttitle\par%
      \ifx\insertsubtitle\@empty%
      \else%
        \vskip0.25em%
        {\usebeamerfont{subtitle}\usebeamercolor[fg]{subtitle}\insertsubtitle\par}%
      \fi%     
    \end{beamercolorbox}%
    \vskip1em\par
    \begin{beamercolorbox}[sep=8pt,center]{author}
      \usebeamerfont{author}\insertauthor
    \end{beamercolorbox}
    \begin{beamercolorbox}[sep=8pt,center]{institute}
      \usebeamerfont{institute}\insertinstitute
    \end{beamercolorbox}
    \begin{beamercolorbox}[sep=8pt,center]{date}
      \usebeamerfont{date}\insertdate
    \end{beamercolorbox}\vskip0.5em
    {\usebeamercolor[fg]{titlegraphic}\inserttitlegraphic\par}
  \end{centering}
  \vfill
}
\makeatother

% insert frame number
\expandafter\def\expandafter\insertshorttitle\expandafter{%
      \insertshorttitle\hfill%
\insertframenumber\,/\,\inserttotalframenumber}

% preset-listing options
\lstset{
  backgroundcolor=\color{white},   
  % choose the background color; 
  % you must add \usepackage{color} or \usepackage{xcolor}
  basicstyle=\footnotesize,
  % the size of the fonts that are used for the code
  breakatwhitespace=false,         
  % sets if automatic breaks should only happen at whitespace
  breaklines=true,                 % sets automatic line breaking
  captionpos=b,                    % sets the caption-position to bottom
  commentstyle=\color{mygreen},    % comment style
  % deletekeywords={...},            
  % if you want to delete keywords from the given language
  extendedchars=true,              
  % lets you use non-ASCII characters; 
  % for 8-bits encodings only, does not work with UTF-8
  frame=single,                    % adds a frame around the code
  keepspaces=true,                 
  % keeps spaces in text, 
  % useful for keeping indentation of code 
  % (possibly needs columns=flexible)
  keywordstyle=\color{blue},       % keyword style
  % morekeywords={*,...},            
  % if you want to add more keywords to the set
  numbers=left,                    
  % where to put the line-numbers; possible values are (none, left, right)
  numbersep=5pt,                   
  % how far the line-numbers are from the code
  numberstyle=\tiny\color{mygray}, 
  % the style that is used for the line-numbers
  rulecolor=\color{black},         
  % if not set, the frame-color may be changed on line-breaks 
  % within not-black text (e.g. comments (green here))
  stepnumber=1,                    
  % the step between two line-numbers. 
  % If it's 1, each line will be numbered
  stringstyle=\color{mymauve},     % string literal style
  tabsize=4,                       % sets default tabsize to 4 spaces
  title=\lstname                   
  % show the filename of files included with \lstinputlisting; 
  % also try caption instead of title
}

% macro for code inclusion
\newcommand{\includecode}[2][c]{
	\lstinputlisting[caption=#2, style=custom#1]{#2}
}
	% nothing to do here
\usepackage[english]{babel}

\usepackage[utf8]{inputenc}

\newcommand{\course}{
	C advanced
}

\author{
	Richard Mörbitz,
	Manuel Thieme
}

\lstset{
	language = C,
	showspaces = false,
	showtabs = false,
	showstringspaces = false,
	escapechar = @,
	belowskip=-1.5em
}

\def\ContinueLineNumber{\lstset{firstnumber=last}}
\def\StartLineAt#1{\lstset{firstnumber=#1}}
\let\numberLineAt\StartLineAt

\makeatletter
\def\beamer@verbatimreadframe{%
	\begingroup%
	\let\do\beamer@makeinnocent\dospecials%
	\count@=127%
	\@whilenum\count@<255 \do{%
		\advance\count@ by 1%
		\catcode\count@=11%
	}%
	\beamer@makeinnocent\^^L% and whatever other special cases
	\beamer@makeinnocent\^^I% <-- PATCH: allows tabs to be written to temp file
	\endlinechar`\^^M \catcode`\^^M=12%
	\@ifnextchar\bgroup{\afterassignment\beamer@specialprocessframefirstline\let\beamer@temp=}{\beamer@processframefirstline}}%
\makeatother % TODO modify this if you have not already done so

% meta-information
\newcommand{\topic}{
	Type qualifiers
}

% nothing to do here
\title{\topic}
\supertitle{\course}
\date{}

% the actual document
\begin{document}

\maketitle

\begin{frame}{Contents}
	\tableofcontents
\end{frame}
\section{Type qualifiers}
\subsection{}

\begin{frame}[fragile]{Extended expressiveness}
To give more information about a variable to the compiler, you can\\
\textit{qualify} its type. There are three type qualifiers in C:\\
\begin{itemize}
	\item \textbf{const}, \textbf{volatile} and \textbf{restrict} (since C99)
\end{itemize}
\bigskip
The syntax is a bit complicated. Normally, a qualifier refers to the type to\\
its left, but the following is also valid (and more common!):
\begin{lstlisting}
const int a;				   /* equal to `int const a` */
\end{lstlisting}
Watch complex types:
\begin{lstlisting}
const int *foo;		/* mutable pointer, constant integer */
int const *foo;							/* same as above */
int * const foo;	/* constant pointer, mutable integer */
int const * const foo; 			  /* everything constant */
\end{lstlisting}
\end{frame}

\begin{frame}[fragile]{\textit{const}}
To prevent all write accesses to a variable, declare its type \textit{const}.\\
\begin{lstlisting}
void f(int *a);
...
const int i = 42;				  /* initialisation is ok */
i = 23; 	   /* error: assignment of read-only variable */
i++;			/* error: increment of read-only variable */
f(&i); /* warning: [...] discards 'const' qualifier [...] */
\end{lstlisting}
The warning above is useful pointer arguments with possible side effects.\\
\bigskip
The initialiser must be a constant expression, but:
\begin{lstlisting}
const int boardWidth; /* depends on runtime input */
...
*(int *) &boardWidth = read_input();
/* This hack is actually undefined behaviour */
\end{lstlisting}
\end{frame}

\begin{frame}[fragile]{\textit{volatile}}
If a variable's value may change between two accesses, we have to\\
prevent the compiler from optimising away subsequent reads or writes.\\
\bigskip
This is mainly used in low-level programming:
\begin{itemize}
	\item Hardware access (memory-mapped I/O)
	\item Threading (another thread modifies a value)
\end{itemize}
\bigskip
Example regarding optimisation:
\begin{columns}
\column{.5\textwidth}
	\begin{lstlisting}
#include <stdio.h>

int main(void) {
    int i = 42;
    printf("%d\n", i);
}
\end{lstlisting}
\column{.5\textwidth}
	\begin{lstlisting}
#include <stdio.h>

int main(void) {
    volatile int i = 42;
    printf("%d\n", i);
}
\end{lstlisting}
\end{columns}
\end{frame}

\begin{frame}[fragile]{\textit{volatile} ctd.}
After compilation with \textit{-O3}:
\begin{columns}
\column{.5\textwidth}
	\begin{lstlisting}[escapeinside={(*}{*)}]
main():
  sub    $0x8,%rsp
  mov    $0x2a,%esi
  mov    $0x400594,%edi
  xor    %eax,%eax
  callq  4003c0 <printf(*@*)plt>
  xor    %eax,%eax
  add    $0x8,%rsp
  retq   
  nopl   0x0(%rax)
\end{lstlisting}
\column{.5\textwidth}
	\begin{lstlisting}[escapeinside={(*}{*)}]
main():
  sub    $0x18,%rsp
  mov    $0x4005a4,%edi
  xor    %eax,%eax
  movl   $0x2a,0xc(%rsp) 
  mov    0xc(%rsp),%esi
  callq  4003c0 <printf(*@*)plt>
  xor    %eax,%eax
  add    $0x18,%rsp
  retq   
  nopw   %cs:0x0(%rax,%rax,1)
  nopl   (%rax)
\end{lstlisting}
\end{columns}
\bigskip
\pause
The compiler could not precompute \textit{i} and pass 42 to \textit{printf} directly.
\end{frame}

\begin{frame}[fragile]{\textit{restrict}}
If a pointer is \textit{restrict}, the compiler assumes that it is the only reference\\
to the targeted memory (no aliases). This enables further optimisation.\\
\bigskip
The most common appearance are function definitions:
\begin{lstlisting}[basicstyle=\scriptsize]
char *strcpy(char * restrict dest, const char * restrict src);
\end{lstlisting}
Note: pointers of different types are always assumed to be no aliases.\\
\bigskip
The compiler is unable to detect if \textit{restrict} pointers actually refer to different memory locations. This has to be ensured by the caller!\\
\bigskip
Non-pointer types cannot be qualified this way.
\end{frame}


% nothing to do from here on
\end{document}
