\documentclass[a4paper,10pt]{article}
%\documentclass[a4paper,10pt]{scrartcl}

\usepackage[utf8]{inputenc}
\usepackage[table]{xcolor}

\usepackage{listings}
\usepackage{makecell}

\lstset{language=C,morekeywords={printf}, keywordstyle=\color{red}}

\title{C Cheatsheet}
\author{}
\date{}


\pdfinfo{%
  /Title    (C Cheatsheet)
  /Author   (Pascal Scholz)
  /Creator  ()
  /Producer ()
  /Subject  ()
  /Keywords ()
}

\begin{document}
\maketitle
\section{Formatstring printf()}

\begin{lstlisting}
printf(%[flags][width][.percision][length]type);
\end{lstlisting} 

\begin{center}
   \rowcolors{1}{white}{yellow}
        \begin{tabular}{|c|l|}
        \hline
            Flag   & Beschreibung\\
        \hline
            -               & Linksausrichten der Ausgabe innerhalb des Platzhalters\\
            +               & Stellt positiven Zahlen ein Vorzeichen voran.\\
            (Leerzeichen)   & Stellt positiven Zahlen ein Leerzeichen voran\\
            0               & Platzhalter wird mit Nullen aufgefüllt (anstatt Leerzeichen)\\
            \verb|#|        & \makecell[l]{ Für G und g Typen: Nachfolgende Nullen werden nicht entfernt.\\
                            Für F, f, e, E, g, G Typen: Output enthält immer einen '.' für Nachkommstellen.\\
                            Für o, x, X Typen: Die Texte 0, 0x, 0X }\\ 
        \hline
        \end{tabular}
    \end{center}
\section{Primitive Datentypen}

\end{document}
